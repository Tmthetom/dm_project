\documentclass[FM,ZP]{tulthesis}
\usepackage[czech]{babel}
\usepackage[utf8]{inputenc}
\usepackage{pdfpages}
\usepackage{graphicx}
\usepackage{tikz}

\TULtitle{Úloha 7 - Počet rozdílných bitů dvou vektorů}{}
\TULprogramme{N2610}{Elektrotechnika a informatika}{Electrical Engineering and Informatics }
\TULbranch{1802T007}{Informační technologie}{Information Technology}
\TULauthor{Bc. Tomáš Moravec}
\TULsupervisor{}
\TULacad{2016/2017}

\begin{document}
\ThesisTitle{CZ}
\renewcommand{\baselinestretch}{1.50}
\setlength\parindent{1.2cm}
\selectfont
	
\begingroup
\renewcommand{\cleardoublepage}{}
\renewcommand{\clearpage}{}
\chapter{Zadání}
\endgroup
Navrhněte synchronní sekvenční obvod se dvěma jednobitovými vstupy X1 a X2. Obvod ukládá sekvenčně tyto dva vstupy do dvou n-bitových vektorů. Číslo n v rozsahu 1 až 8 je možné kdykoliv měnit opakovaných stiskem tlačítka (po hodnotě 8 následuje cyklicky hodnota 1). Výstupem je číslo udávající počet rozdílných bitů obou vektorů.
	
\begingroup
\renewcommand{\cleardoublepage}{}
\renewcommand{\clearpage}{}
\newpage
\chapter{Řešení úlohy}
\endgroup
Úloha byla na základě dporučení cvičícího rozdělena do bloků a následně řešena po jednotlivých částech.Výsledné bloky umožňují čtení vstupů, porovnávání hodnot a také výsledku zobrazení na segmentovkách.

\section{Pulse}
Tento blok má za vstup tlačítko a řeší, aby stiknutým tlačítkem byl vyslán pouze jediný pulz do klopného obvodu. Řešení bylo zhotoveno přes stavový automat.

\section{Memory}
Memory blok je zhotoven z dvoustupého klopného obvodu D a multiplexoru. Pokud není v multiplexoru adresní vstup v 1, tak se hodnota v obvodu D nemění, jinak je načtena hodnota nová.

\section{Machine}
Jedná se o automat, který má 8 stavů. V každém stavu je aktivní jiný výstup a je tak detekováno, jaké paměťové bloky se mají porovnat a sečíst. Z jednoho stavu do druhého se lze dostat jen při aktivním vstupu, jinak zůstavá v aktuálním stavu.

\section{Sum}
Úkolem tohoto velice jednoduchého bloku je zjištění počtu rozdílných biitů a to tak, že pomocí XOR získá stav 0 nebo 1, výsledné hodnoty se následně sečtou pomocí převodu bity na integer, integer na unsignet integer, následně proběhne sečtění a výsledek je převeden na std-logic-vector.
	
\section{Decoder-segm}
Blok přijímá jedinou vstupní hodnotu a jeho výstupem je zobrazení čísla na sedmisegmentový displej.

\end{document}